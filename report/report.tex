\documentclass{article}
\usepackage[utf8]{inputenc}
\usepackage[english]{babel}
\usepackage{amsmath}
\usepackage{ulem}
\usepackage{palatino}
\linespread{1.05}
\title{Advanced Programming Assignment 2 - Number Sets in Prolog}
\author{Philip Munksgaard \& Ronni Elken Lindsgaard}
\newcommand{\prolog}[1]{\texttt{#1}}
\newcommand{\datatype}[1]{\underline{\textbf{#1}}}
\newcommand{\direction}[1]{\textbf{#1}}
\begin{document}
\maketitle
\section{Code documentation}
We have implemented the code in a prolog module file called sets.pl. The
predicate \prolog{less/2} recursively checks whether $\prolog{X} < \prolog{Y}$.
\prolog{checkset/1} lifts off the first to elements in a list and checks that
the first is the lesser one. The second element is pushed onto the list again
and recursion is used to run through all elements of the list.

The \prolog{ismember/3} predicate mainly uses pattern matching in order to
decide the result. The logic for each line is described here
\begin{description}
  \item If the list is empty, the result should only be true if the third
    parameter is \direction{no}, and if the element we are looking for is a natural number.
  \item If the element in the list matches the element we are looking for, we
    make sure that the element is also a natural number.
  \item If the head of the list is bigger than the element we are
    looking for, so are all the rest of the elements of the list,
    since the head is the smallest one (given that the list is a
    proper representation of a set of natural numbers), thus the third
    parameter should be \direction{no}.
  \item If the head of the list is smaller than the element we are
    looking for, the element still might be in the list, so we call
    \prolog{ismember/3} recursively on the rest of the list.
\end{description}

If one of the sets $s_1$ or $s_2$ are empty, then \prolog{union/3} returns the
non-empty set. We believe it is bad practice to return a illegitimate set and
therefore make a call to \prolog{checkset/1}.
We recurse through the two lists and makes sure that \prolog{XS} always contain
the set with the lesser numbers.

\prolog{intersection/3} works just like \prolog{union/3} except that we only
match subjects in $t_3$ if they are present in both $t_1$ and $t_2$.

\section{Assessment of our solution}
The program matches the requirements and we believe that we have implemented a
good solution. We have made excessive testing with plunit that covers a wide
range of possibilities that should succeed and also what should fail,
specifically testing the examples from the assignment text. We believe
that we have made reasonable and well founded judgment for behavior that was not
specified in the assignment.
Our code is comprehensible and clear. Due to the small codebase, and the
documentation above, we have omitted inline comments.
\end{document}
