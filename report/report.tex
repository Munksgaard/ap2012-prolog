\documentclass{article}
\usepackage[utf8]{inputenc}
\usepackage[english]{babel}
\usepackage{amsmath}
\usepackage{ulem}
\usepackage{palatino}
\linespread{1.05}
\title{Advanced Programming Assignment 2 - Number Sets in Prolog}
\author{Philip Munksgaard \& Ronni Elken Lindsgaard}
\newcommand{\prolog}[1]{\texttt{#1}}
\newcommand{\datatype}[1]{\underline{\textbf{#1}}}
\newcommand{\direction}[1]{\textbf{#1}}
\begin{document}
\maketitle
\section{Code documentation}
We have implemented the code in a prolog module file called
sets.pl. The predicate
\prolog{less/2} recursively checks whether $\prolog{X} < \prolog{Y}$.
\prolog{checkset/1} lifts off the first to elements in a list and
checks that the first is the lesser one. The second element is pushed
onto the list again and recursion is used to run through all elements
of the list.

The \prolog{ismember/3} predicate assumes that we get a proper natural
number set as defined in the assigment, and mainly uses pattern
matching in order to decide the result. The logic for each rule is
described here
\begin{enumerate}
  \item If the list is empty, the result should only be true if the third
    parameter is \direction{no}.
  \item If the element in the list matches the element we are looking for, we should return \direction{yes}.
  \item If the head of the list is bigger than the element we are
    looking for, so are all the rest of the elements of the list,
    since the head is the smallest one (given that the list is a
    proper representation of a set of natural numbers), thus the third
    parameter should be \direction{no}.
  \item If the head of the list is smaller than the element we are
    looking for, the element still might be in the list, so we call
    \prolog{ismember/3} recursively on the rest of the list.
\end{enumerate}

If one of the sets $s_1$ or $s_2$ are empty, then \prolog{union/3} returns the
non-empty set. We believe it is bad practice to return a illegitimate set and
therefore make a call to \prolog{checkset/1}.
We recurse through the two lists and makes sure that \prolog{XS} always contain
the set with the lesser numbers.

\prolog{intersection/3} works just like \prolog{union/3} except that we only
match subjects in $t_3$ if they are present in both $t_1$ and $t_2$.

\subsection{The behavior of \prolog{ismember(N, [s(z), s(s(s(z)))], A).}}
The second rule is matched at first, the head \prolog{s(z)} is found and
\prolog{A = yes}. Then prolog tries to find all possibilites for \prolog{N}. The
query always succeeds since it is always possible to find a suitable \prolog{N}
where \prolog{A = no}. When \prolog{N = s(s(s(z)))} the fourth rule is matched
at first which results in a recursive call that finally results in matching of
the second rules which makes \prolog{A = yes}.

\section{Assessment of our solution}
The program matches the requirements and we believe that we have implemented a
good solution. We have made excessive testing with plunit that covers a wide
range of possibilities that should succeed and also what should fail,
specifically testing the examples from the assignment text. We believe
that we have made reasonable and well founded judgment for behavior that was not
specified in the assignment.
Our code is comprehensible and clear. Due to the small codebase, and the
documentation above, we have omitted inline comments.
\end{document}
